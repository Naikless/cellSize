%% SPRACHE %%%%%%%%%%%%%%%%%%%%%%%%%%%%%%%%%%%%%%%%%%%%%%%%%
%--------------- DEUTSCH
%\usepackage[ngerman]{babel} 
%\usepackage[T1]{fontenc}  
%\usepackage[latin1]{inputenc} 
% Wenn Sie Fehler wie Textcurrency Unavailable o.ä. erhalten, speichern Sie die Datei unter einer anderen Formatierung ab (z.B. ANSI)

%--------------- ENGLISCH
\usepackage[USenglish]{babel}
\usepackage[T1]{fontenc}
\usepackage[latin1]{inputenc}

%% AUSSEHEN %%%%%%%%%%%%%%%%%%%%%%%%%%%%%%%%%%%%%%%%%%%%%%%%
\usepackage{bera}
\usepackage[charter]{mathdesign} % Schriftart
%\usepackage{chngcntr} % Change Counter Package
\usepackage{abstract} % Abstract benutzen
\usepackage{microtype} % verbessert Textverteilung
\usepackage{multicol} % Mehrspaltige Abschnitte
\setlength{\columnsep}{.5cm}
\usepackage[symbol,perpage]{footmisc}
\setfnsymbol{lamport}

%% GRAFIKEN %%%%%%%%%%%%%%%%%%%%%%%%%%%%%%%%%%%%%%%%%%%%%%%%
\usepackage{graphicx} % Einbinden von Grafiken ( .jpg  .png  .pdf  .mps)
%--------------- TIKZ Umgebung (zum Zeichnen von Grafiken in Latex
\usepackage{epstopdf}
\usepackage[usenames,dvipsnames]{xcolor}
\usepackage{tikz}
\usetikzlibrary{arrows,shapes,positioning,shadows,calc,intersections}
\usepackage{pgfplots}
\pgfplotsset{compat=1.8}
\usepackage{marginnote}
\usepackage{caption}


%% Math Packages %%%%%%%%%%%%%%%%%%%%%%%%%%%%%%%%%%%%%%%%%%%%
\usepackage{amsmath}
\usepackage{amsthm}
%\usepackage{amsfonts}

%% NOMENKLATUR %%%%%%%%%%%%%%%%%%%%%%%%%%%%%%%%%%%%%%%%%%%%%%
\usepackage{nomencl}
\let\abk\nomenclature
\makenomenclature
\usepackage{etoolbox}
\patchcmd{\thenomenclature}{\section*{\nomname}}
    {\begin{multicols}{2}[\section*{\nomname}]}{}{}
\patchcmd{\endthenomenclature}{\endlist}{\endlist\end{multicols}}{}{}

%% MATLAB CODE %%%%%%%%%%%%%%%%%%%%%%%%%%%%%%%%%%%%%%%%%%%%%%
\usepackage[framed]{matlab-prettifier}
\lstset{
  style              = Matlab-editor,
  basicstyle         = \scriptsize\ttfamily
  }

%% BIBLIOGRAPHIE %%%%%%%%%%%%%%%%%%%%%%%%%%%%%%%%%%%%%%%%%%%%
\usepackage{etoolbox}
\usepackage[square,numbers]{natbib}
\makeatletter
%\renewcommand*{\NAT@nmfmt}[1]{\textsc{#1}}
%\def\NAT@nmfmt#1{\textsc{#1}}
\patchcmd{\NAT@test}{\else\NAT@nm}{\else\NAT@nmfmt{\NAT@nm}}{}{}
\let\NAT@up\scshape
\makeatother
%\renewcommand{\citenumfont}[1]{\textsc{#1}}


%% ZEILENABSTÄNDE %%%%%%%%%%%%%%%%%%%%%%%%%%%%%%%%%%%%%%%%%%%
%\usepackage{setspace}
%\singlespacing        %% 1-spacing (default)
%\onehalfspacing       %% 1,5-spacing
%\doublespacing        %% 2-spacing

% End Paragraph Command
\newcommand{\parend}{\\\hspace*{\fill}} % \hspace*{\fill} füllt die durch \\ von Latex erwartete nächste Zeile und verhindert so die "`zu leere Box"'

%% HYPERREF-PACKAGE %%%%%%%%%%%%%%%%%%%%%%%%%%%%%%%%%%%%%%%%%
\usepackage[colorlinks=true, urlcolor=black, linkcolor=black, citecolor=black, bookmarksnumbered=true, bookmarksopenlevel=0, pdfpagelabels=false]{hyperref} % URL
\def\UrlBreaks{\do\-\do\_\do\+\do\/\do\a\do\b\do\c\do\d\do\e\do\f\do\g\do\h\do\i\do\j\do\k\do\l%
\do\m\do\n\do\o\do\p\do\q\do\r\do\s\do\t\do\u\do\v\do\w\do\x\do\y\do\z\do\0%
\do\1\do\2\do\3\do\4\do\5\do\6\do\7\do\8\do\9}

